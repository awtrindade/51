\chapter{Discussion and conclusion}
\label{s:conclusion}

Although languages around the world exhibit a high variation in how
they describe colours, most of the research on colour categorisation
and colour naming has focussed on the use of single colour terms. In a
cross-cultural study, \cite{alvarado02modifying} study how modifiers
are used in English and Vietnamese. Their results show that both
languages use different naming strategies. English speakers employ a
greater variety of basic colour terms whereas Vietnamese speakers
prefer to use modifiers over single-word basic terms. These
observations reflect that different languages use different strategies 
to describe colours, which reflect the general strategy used in a language.
This, in turn, casts serious doubts on the results obtained by
the research paradigm which focusses solely on the use of basic colour
terms, which has been the mainstream paradigm in anthropological
research up until today \citep{kay10world}.

The main goal of this book is to propose a similar shift in the
computational models for colour in the domain of artificial language
evolution, which so far have always been limited to single colour terms. This
shift requires richer semantics to model the meaning of more complex
colour descriptions and richer syntax to express these meanings in
language. Although these models are more complex, great care 
should be taken to ensure they are still computationally tractable.

\section{Contributions}
\label{s:contributions}

\subsection{Identification of language strategies}

The first contribution of this book is the identification of
different language strategies based on the way colours are described
in unconstrained colour naming studies, anthropological research and
psychology. This identification allows to establish clear goals and
defines a clear scope for this book. I have chosen to focus on four
different language strategies: the basic colour strategy, the
graded membership strategy, the category combination
  strategy and the basic modification strategy.

The basic colour strategy limits language users to use a single
term to describe a colour sensation. This strategy has received by far
the most attention in colour research. Examples in English are
\textit{green} and \textit{blue}. In the graded membership strategy
language users are able to specify the degree of membership of a
particular colour sample to a basic colour category, as for example in
\textit{very green} or \textit{greenish}. The category combination
  strategy allows language users to combine more than one category to
describe a colour in more detail, like in \textit{blue-green}. This
strategy can be extended with the graded membership strategy,
which would allow to specify how well the colour sample represents one
of the constituent categories, as in \textit{reddish brown}. The final
language strategy I consider in this book is the basic
  modification strategy which allows a language user to modulate
particular subdimensions of a colour, such as the lightness modifiers
\textit{light} and \textit{dark} and the chromaticity modifiers \textit{pale} and
\textit{bright}.

\subsection{Operationalisation of language strategies}

The operationalisation of the basic colour strategy draws on
the previous models proposed in the field of artificial language
evolution for the domain of colour \citep{steels05coordinating}. These
models make use of perceptual colour spaces in which each colour
sensation is represented by a unique point. In one of these models
\citep{belpaeme05explaining}, colour categories are represented by a
single point or prototype, in the same space which is the best
representative of that colour category \citep{rosch73natural}. Colour
samples are categorised as the colour category of which the prototype
is most similar to the perceived colour sample. Naming these samples
is achieved through an associative network that assigns symbols to
these categories.

The graded membership strategy is implemented using the
categorisation principles of the basic colour strategy, but
additionally the agents can express the distance to the prototype of
that category. This ability is operationalised through a
categorisation process based on the similarity between the prototype
of the colour category and the colour sample that needs to be
described. This second categorisation process is based on a nearest
neighbour classification algorithm which uses prototypical similarity
values as categories. These categories are named using symbols which
are stored in an associative network.

The category combination strategy also builds on the
categorisation process of the basic colour strategy. Before
applying a second categorisation process based on colour, the set of
colour categories needs to be transformed. The transformation I have
implemented is moving each of the categories in the set in the
direction of the category that was used during the first
categorisation process. This allows the language users to re-use the
same set of colour categories to further specify the subregions of the
basic colour category. This strategy can be extended with a third
categorisation process based on similarity, similar to the one
introduced for the previous strategy.

Finally, the basic modification strategy is
operationalised similar to the category combination
strategy. The main difference is that basic modifiers are only
specified in some dimensions of the colour domain, so a different
transformation method is required. I proposed a scaling method which
places each of the modifiers on the borders of the original colour
category.

I have also proposed naming benchmarks to determine the performance of each
of the operationalised strategies. These benchmarks are based on
psychological studies which identified the colour samples that
represent a certain category or colour description best. Some of these
studies also reported a set of consensus samples, which is defined as
the subset of all samples that are named consistently by all
participants in one experiment. The naming benchmark consists of
naming all these reported samples and verifying whether the resulting
descriptions correspond to those reported in literature. All
operationalisations reached more than satisfactory results on these
benchmarks, indicating the validity of the proposed
operationalisations.

\subsection{Self-organisation of language systems}

I have introduced the learning operators for the basic colour
  strategy and the graded membership strategy. The adoption
and alignment operators allow an agent to acquire colour categories and
membership categories from another agent. The performance of these
operators have been evaluated in an acquisition experiment in which a
learning agent needs to acquire the language system of a teaching
agent. Additionally, the invention operator allows agents to establish
their own language system based on these strategies, which has been
evaluated in a formation experiment.

Furthermore, I have used the basic colour strategy to study the
impact of statistical distributions of colour in the environment, the
use of language, and embodiment on the resulting colour category
systems.

Some researchers claim the statistical distribution of colour has a
significant impact on the colour category systems that are formed
within them \citep{yendrikhovskij01computational}. To verify this
hypothesis, I have compared three artificial environments: one based
on natural scenes, one based on urban scenes and one in which the
chance of encountering a colour was uniform. I used an individual
learning algorithm in which agents need to learn to categorise these
worlds and compared the resulting category systems to English
colour categories. Although most of the previously reported
correlations were also present in the uniform data set, a small yet
positive impact of the statistical distribution of colour on this
correlation was present in the natural and urban data sets.

In a second experiment I have compared individual learning to social
learning in a population of agents using language as mediator. I
compared the resulting colour category systems to similar data of 110
natural pre-industrial languages around the world
\citep{kay10world}. The study of these natural languages revealed
some universal tendencies in the position of colour categories. The
impact of language on the similarity to these tendencies was found to
be positive but rather low.

In a third experiment, I have studied the impact of embodiment on the
formation of colour category systems. In most of the previous
experiments in the domain of colour, the contexts that were presented
to the agents were generated from a uniform set of colour samples. In
an embodied setting, colourful objects are presented to the vision
system of robots. In this setting, the perceived colours are
highly structured and are all situated around the actual colour of
these objects. This structure helps the agents to align
their colour category systems. But embodiment also introduces an
additional factor that hinders communication: as each robot perceives
a scene from its own perspective, the colour sensations will never be
identical. I have shown that the proposed algorithms are robust enough
to overcome this problem.

\subsection{Evolution of language strategies}

Language strategies evolve over time. If one looks at the history of
the basic colour strategy for English, an interesting meaning shift of
the basic colour categories occurred: they simultaneously shifted from
a brightness sense to a hue sense. I have presented a model to study
this phenomenon in which linguistic selection occurs at two levels: at
the level of linguistic items that are used by the agents in the
interactions and at the level of language strategies. This additional
level can be used when the language system needs to be expanded
without endangering the systematicity of the system, but it can also
be used to lower the risk of misunderstandings in more creative
language use. The presented model exhibits similar dynamics to the
ones observed in the history of English.

\subsection{Compositional semantics and language}

Throughout the book, great care was taken to ensure a high level of
compositionality, both at the level of semantics and at the level of
language. This compositionality increases the level of potential
re-use of certain parts of knowledge, which results in a higher level
of expressivity without the cost of establishing new agreements in the
population. I have introduced repair strategies that allow agents to
invent and coordinate a compositional language for the compositional
semantics they are using. I have shown experimental results in which a
population of agents invents and aligns its compositional language.

\section{Discussion}

\subsection{Tractability}

Tractability of the semantic templates is mainly an issue during the
conceptualisation process in which the speaker has to find the
appropriate semantic template and categories to discriminate a colour
in a specific context. During interpretation tractability is less of
an issue as the hearer should be able to understand which categories and
semantic template are used. 

The semantic templates can be implemented in many different ways. It
is for example possible to implement the category combination
  strategy by generating all possible combinations of two basic
colour categories and choosing the combination that describes the
target entity (or set) best. This would however be inefficient as 
the combinatorial process would generate a huge number of colour
categories. The number of categories can be restricted by
incorporating the context of the current interaction. In the semantics
I propose in this book, the first filtering operation limits the
number of base categories that are relevant in the current context. If
the context, for example, consists of two shades of green and one purple
colour, it ensures only the green and the purple category would be
considered as base categories, and would for example exclude yellow as
relevant base category. Incorporating the context at each possible
step in the semantic template increases its tractability during
conceptualisation.

\subsection{Compositionality}

In order to allow a population of agents to bootstrap a complex
language, it is crucial to ensure it can reach this level by starting
from simpler languages that can be made more complex in a stepwise
fashion. In the proposed semantic templates, this is achieved by
implementing semantic primitives in such a way that they can be re-used in a
wide range of semantic templates. For example, the \textsc{Filter-by-Colour}
primitive can be used both in the basic colour
  strategy and in the category combination strategy in
which it is used twice. This re-use in turn ensures that more complex
templates can be conceived as minor extensions of simpler ones.

As these semantic templates are used in linguistic interactions, the
linguistic rules that are used to express these templates should
reflect the same compositionality. This allows an agent to re-use
much of the previously established language when they need to express
a more complex semantic template and it also allows agents to establish only
small parts of the language system at the time. By implementing the
semantic and syntactic templates in a compositional way, the agents
can gradually build up a complex language system.

\subsection{Flexiblity}

Although compositionality allows agents to bootstrap a complex
language from simpler languages, the actual expressive power depends
on the number of ways the semantic primitives can be combined in new semantic templates.
The proposed semantic primitives
are designed to reflect a high level of flexibility and re-use. In the
basic modification strategy case I propose a
transformation of the set of modifiers. But depending on the contexts
the agents have to communicate about, this transformation
might not be needed. Another example would be the use of the modifiers
in the different language strategies. The proposed templates for the
graded membership strategy could also be used with the set of
lightness modifiers by replacing the primitive that retrieves the set
of categories from the ontology of the agent.

\subsection{Generality}

Although the proposed semantic and syntactic templates are limited to
the domain of colour, they are considered to be general and therefore can be
applied to examples from other domains. 
For example, the \textsc{Filter-by-Membership} primitive can be used to
represent the graded membership in any domain, such as time
(e.g. \textit{very soon}) or space (e.g. \textit{very near}). The proposed
transformation \textsc{Scale-Category-Set-to-Category} of the set of
categories can also be used to represent the transformation required
to describe the colour of a more specific concept, such as \emph{wine}. The colour
categories used to describe wine are different from the their original
meaning as the colour of \textit{white wine} for example, is in fact more
yellowish than white, indicating that a transformation might be required.

The main principles behind the proposed syntactic templates and the repair
strategies for learning these templates are also considered to be
general and applicable to any kind of semantic constraint
network. The link feature will always be required to ensure the
different semantic primitives are combined in the right order and the
re-use of syntactic categories can be used to restrict the number of applicable
grammatical rules during the parsing of utterances.

\subsection{Related models and approaches}

\subsubsection{Models of colour naming}

Most other computational models for the domain of colour focus on the
naming of certain colour samples. Some of them store the names and
related foci based on predefined colour dictionaries
\citep{mojsilovic05computational}, others use the colour information
of images found on the web \citep{vandewijer07learning}, while a
third approach uses data from psychological experiments
\citep{menegaz07discrete}. Naming a colour sample boils down to
finding the focus that is most similar to a particular colour sample.

The method proposed in this book is more general, as it fully
embraces the idea of the living nature of spoken languages. By using
richer semantics, the proposed model is also capable of predicting
certain compound colour terms that are not part of the predefined
lexicon. By enabling the adoption and alignment operators, the
proposed model could also acquire the meaning of new colour
descriptions and store them for later use.

\subsubsection{Fuzzy sets}

Fuzzy set theory is an often pursued approach to the semantics of the
domain of colour. It stipulates that colour samples are rarely a
complete member of a single category, but are better approached by
fuzzy members that are part of several categories. This approach might
lead to the unsatisfactory treatment of some of the basic colour
categories \citep{kay78linguistic}. It is also unclear how to scale it
to conceptual spaces with a dimensionality higher than one. One
solution to this problem might be to artificially divide some
dimensions in planes \citep{benavente08parametric}.

This approach seems particularly promising for the category
  combination approach, but it is less clear how it could be extended
to suit the graded membership approach. Some suggestions have
been made on how to deal with adverbs \citep{hersh76fuzzy}, but these
seem unsatisfactory in explaining why the use adverbs should be
preferred in language.

\subsubsection{Conceptual spaces}

The theory of conceptual spaces probably provides the most similar
approach to the semantics proposed in this book. The approach for
the basic colour strategy is based on this theory
\citep{gardenfors04conceptual}. How concepts should be combined
however is only loosely defined. The work in this book tries to
formalise some possible transformations that could fit this bill.

\subsubsection{Vantage theory}

Another approach to semantics that resonates with the proposed
semantics is the vantage theory. In this theory, the set of colour
primaries are some reference points that can be used to construct more
complex colour categories. Each colour category is defined as a series
of processes based on similarity and distinctiveness. These processes
can be either based on hue or on brightness. The colour categories can
be compositional in the sense that they incorporate more than one
hue. Overall, it allows for a wide variety on how colour
categorisation could take place. Although this theory focusses on the
organisation of a single category, it is clear that it shares some
ideas with the proposed semantics \citep{maclaury92brightness,
  maclaury02introducing}.

\section{Possible applications}
\label{s:possible-applications}

One of the possible applications of the proposed methodology is the
development of interactive tutoring systems. These systems can be used
to tutor colour systems, e.g. the Russian colour system,
to human students. In each interaction, a set of colour samples is
presented to the student. The system describes one of the samples to
the student and the student has to guess which colour sample the tutor
intended. In the background, the system constructs a model of the
student by applying the adoption and alignment operators. This will
allow the system to predict how the student will describe each sample
and, if this mismatches with the target description the system has in
mind, to draw the attention of the student on that particular sample.

But the roles in the tutoring system can also be reversed. The human
now becomes the tutor and the system learns the language of the human
tutor using the adoption and alignment operators. In each interaction
the system generates a set of colour samples and the tutor types in a
description of one of the colours. The goal of the system is now to
guess which colour sample is intended by the human tutor. When the
human tutor is satisfied by the performance of the system, the system
stores all linguistic and conceptual knowledge it has built up. This
knowledge can later be used to tutor the stored language system to
another human student.

A prototype of such a tutoring system based on the basic colour
  strategy has been implemented and could be extended to more complex
strategies. This prototype can be used as the starting point for  other tutoring
systems which focus on aspects of languages that are hard to learn,
such as aspect in Russian or the use of posture verbs (\textit{staan}, \textit{zitten} and
\textit{liggen}) in Dutch.

\section{Future work}
\label{s:future-work}

The most direct and exciting way to extend the current work is to
implement an experiment that focusses on the origins of language
strategies. This work could start from the repair strategies presented
in this book. The corresponding experiment has to be extended
in order to allow agents to generate their own semantic templates. 
It would be interesting to
see what kind of templates arise and what mechanisms might enable
agents to align the templates they are using.

Although the scope of this book is limited to the domain of colour,
it would be interesting to apply the proposed semantic to other
domains, such as space or time. This could also include naming the
colour of more general concepts (such as \emph{wine}) or to use the typical
colour of a concept to describe a colour (as in \textit{blood red}).