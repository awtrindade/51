%% -*- coding:utf-8 -*-


%%%%%%%%%%%%%%%%%%%%%%%%%%%%%%%%%%%%%%%%%%%%%%%%%%%%
%%%                                              %%%
%%%     Language Science Press Master File       %%%
%%%         follow the instructions below        %%%
%%%                                              %%%
%%%%%%%%%%%%%%%%%%%%%%%%%%%%%%%%%%%%%%%%%%%%%%%%%%%%

% please fill in some information in the following lines as soon
% as you have it
% Everything following a % is ignored
% Some lines start with %. Remove the % to include them

\documentclass[number=??                 %replace by your number in series
                ,series=dummyseries,     % Choose series abbreviation as appropriate
                ,isbn=xxx-x-xxxxxx-xx-x, %add your isbn here
                ,url=http://langsci-press.org/catalog/book/0,  %change to the running number of your book
	        ,output=long             % long|short|inprep              
	        %,blackandwhite
	        %,smallfont
	        ,draftmode  
		  ]{langsci}    



%%%%%%%%%%%%%%%%%%%%%%%%%%%%%%%%%%%%%%%%%%%%%%%%%%%%
%%%                                              %%%
%%%            General Setup                     %%%
%%%         no need to change this               %%%
%%%                                              %%%
%%%%%%%%%%%%%%%%%%%%%%%%%%%%%%%%%%%%%%%%%%%%%%%%%%%%

% \hypersetup{pdfdisplaydoctitle=true} % This should all go to *cls
% \usepackage{tabularx}
% \selectlanguage{USenglish} 
 

%%%%%%%%%%%%%%%%%%%%%%%%%%%%%%%%%%%%%%%%%%%%%%%%%%%%
%%%                                              %%%
%%%           Examples                           %%%
%%%                                              %%%
%%%%%%%%%%%%%%%%%%%%%%%%%%%%%%%%%%%%%%%%%%%%%%%%%%%%
% remove the percentage signs in the following lines
% if your book makes use of linguistic examples

\usepackage{lsp-gb4e} 
%% to add additional information to the right of examples, uncomment the following line
% \usepackage{jambox}
%% if you want the source line of examples to be in italics, uncomment the following line
% \def\exfont{\it}

%%%%%%%%%%%%%%%%%%%%%%%%%%%%%%%%%%%%%%%%%%%%%%%%%%%%
%%%                                              %%%
%%%          Trees                               %%%
%%%                                              %%%
%%%%%%%%%%%%%%%%%%%%%%%%%%%%%%%%%%%%%%%%%%%%%%%%%%%%

% For trees, uncomment the following lines
% \usepackage{tikz-qtree}
% % has strange side effects
% %\tikzset{every tree node/.style={align=left, anchor=north}}
% \tikzset{every roof node/.append style={inner sep=0.1pt,text height=2ex,text depth=0.3ex}}

%%%%%%%%%%%%%%%%%%%%%%%%%%%%%%%%%%%%%%%%%%%%%%%%%%%%
%%%                                              %%%
%%%      Optimality Theory                       %%%
%%%                                              %%%
%%%%%%%%%%%%%%%%%%%%%%%%%%%%%%%%%%%%%%%%%%%%%%%%%%%%
% If you are using OT, uncomment the following lines      
% % OT pointing hand
% \usepackage{pifont}
% \newcommand{\hand}{\ding{43}}
% % OT tableaux                                                
% \usepackage{pstricks,colortab}    

%%%%%%%%%%%%%%%%%%%%%%%%%%%%%%%%%%%%%%%%%%%%%%%%%%%%
%%%                                              %%%
%%%       Attribute Value Matrices               %%%
%%%                                              %%%
%%%%%%%%%%%%%%%%%%%%%%%%%%%%%%%%%%%%%%%%%%%%%%%%%%%%
%If you are using Attribute-Value-Matrices, uncomment the following lines 
% \usepackage{lsp-avm}
% \usepackage{avm}
% \avmfont{\sc} 
% \avmvalfont{\it} 
% % command to fontify the type values of an avm 
% \newcommand{\tpv}[1]{{\avmjvalfont #1}} 
% % command to fontify the type of an avm and avmspan it
% \newcommand{\tp}[1]{\avmspan{\tpv{#1}}}


%%%%%%%%%%%%%%%%%%%%%%%%%%%%%%%%%%%%%%%%%%%%%%%%%%%%
%%%                                              %%%
%%%     Discourse Representation Structures      %%%
%%%                                              %%%
%%%%%%%%%%%%%%%%%%%%%%%%%%%%%%%%%%%%%%%%%%%%%%%%%%%%
% DRS package by Alexis Dimitriadis
% \usepackage{drs}

%%%%%%%%%%%%%%%%%%%%%%%%%%%%%%%%%%%%%%%%%%%%%%%%%%%%
%%%                                              %%%
%%%            Chinese Japanese Korean           %%%
%%%                                              %%%
%%%%%%%%%%%%%%%%%%%%%%%%%%%%%%%%%%%%%%%%%%%%%%%%%%%%

% For Chinese characters, uncomment the following lines
% \usepackage[indentfirst=false]{xeCJK}
% \setCJKmainfont{SimSun}

%%%%%%%%%%%%%%%%%%%%%%%%%%%%%%%%%%%%%%%%%%%%%%%%%%%%
%%%                                              %%%
%%%               Arabic / Persian               %%%
%%%                                              %%%
%%%%%%%%%%%%%%%%%%%%%%%%%%%%%%%%%%%%%%%%%%%%%%%%%%%%

% for bidirectional text and support for Arabic/Persian, uncomment the following lines
%% \usepackage{fontspec}
% \newfontfamily\Parsifont[Script=Arabic]{XB Niloofar}
% %\usepackage{bidi}
% \usepackage{lsp-bidi}
% \newcommand{\PRL}[1]{\RL{\Parsifont #1}}
% %\TeXXeTOff
 

 
%%%%%%%%%%%%%%%%%%%%%%%%%%%%%%%%%%%%%%%%%%%%%%%%%%%%
%%%                                              %%%
%%%          additional packages                 %%%
%%%                                              %%%
%%%%%%%%%%%%%%%%%%%%%%%%%%%%%%%%%%%%%%%%%%%%%%%%%%%%

% put all additional commands you need in the 
% following files

\usepackage{localmetadata}
\usepackage{localpackages}
\usepackage{localhyphenation}
\usepackage{localcommands}

%%%%%%%%%%%%%%%%%%%%%%%%%%%%%%%%%%%%%%%%%%%%%%%%%%%%
%%%                                              %%%
%%%               END PREAMBLE                   %%%
%%%                                              %%%
%%%%%%%%%%%%%%%%%%%%%%%%%%%%%%%%%%%%%%%%%%%%%%%%%%%%
% -----------------------------------------------%%%
%%%%%%%%%%%%%%%%%%%%%%%%%%%%%%%%%%%%%%%%%%%%%%%%%%%%
%%%                                              %%%
%%%             BEGIN DOCUMENT                   %%%
%%%                                              %%%
%%%%%%%%%%%%%%%%%%%%%%%%%%%%%%%%%%%%%%%%%%%%%%%%%%%%      
\begin{document}       
%%%%%%%%%%%%%%%%%%%%%%%%%%%%%%%%%%%%%%%%%%%%%%%%%%%%
%%%                                              %%%
%%%             Frontmatter                      %%%
%%%                                              %%%
%%%%%%%%%%%%%%%%%%%%%%%%%%%%%%%%%%%%%%%%%%%%%%%%%%%%        
\maketitle                
\frontmatter
% %% uncomment if you have preface and/or acknowledgements
% \chapter*{Preface} 
% \addchap{Preface}

Although languages around the world display an overwhelming variety in
ways to describe colours, most of the research in the domain of colour
has focussed on the use of single colour terms. This approach has
allowed researchers in a wide range of fields to tackle interesting
questions, such as the extent to which colour categories are innate or
learned. In the field of artificial language evolution, the focus on
single colour terms has enabled researchers to build computational
models in which populations of linguistic agents can construct and
coordinate their own colour category system so that they become
successful in communication.

A few descriptive studies report on describing colours beyond the
restriction of using a single colour term. The results of these
studies seem conclusive: only a small minority (around 15\%) of all
colour samples would be described using a single colour term. Most
samples are described using more elaborate expressions, for example by
using modifiers or combinations of colour terms.

In this book, I show how the current models in artificial language
evolution can be extended to allow for richer descriptions of colour
samples. In order to do so, I deploy two powerful formalisms that have
been developed to support this kind of experiments: Incremental
Recruitment Language (IRL) to represent the semantics, or meaning, of
linguistic utterances and Fluid Construction Grammar (FCG) to
transform these meanings into linguistic utterances and back.

Four different language strategies are explored: the basic colour
strategy\linebreak (``blue''), the graded membership strategy (``greenish''),
the category combination strategy (``blue-green'') and the basic
modification strategy (``dark blue''). Each of these strategies is
realised in different languages around the world and some studies
reported on the most prototypical colour samples that are associated
with these expressions. For each strategy, I propose a semantic
template which captures the general cognitive operations required to
use that particular strategy and syntactic templates which represent
general grammatical rules that can express semantic templates in
language. I pursue a compositional approach, focussing on the re-use
of semantic primitives and syntactic templates as much as possible. I
show that more complicated language strategies can be conceived as
minor extensions of the basic colour strategy and that only a few
syntactic templates suffice to express all these strategies. Once
these strategies have been operationalised, I compare their naming
behaviour to human data reported in the literature. The performance of the
strategies can be compared in a baseline experiment in which simulated
language users engage in linguistic interactions, the
difficulty of which is carefully controlled.

The implementation of a language strategy can be completed by adding
learning operators which allow an agent to pick up the language system
of another agent and to extend the current language system whenever
the communicative need arises. The performance of these operators is
tested in an acquisition and a formation experiment. In an acquisition
experiment, one agent knows a predefined language system and acts as a
teacher. The goal of the learner agent is to acquire the predefined
language system and to become as successful in communication as two
agents which share perfect knowledge of the predefined language
system. In a formation experiment, a population of agents need to
invent and coordinate their own language system based on a particular
language strategy. I present results for both the basic colour
strategy and the graded membership strategy.

Once the implementation of a language strategy is completed, in-depth
studies can be carried out. I show the results of three different
studies using the basic colour strategy: (a) the positive impact of
the statistical distribution of colours in the environment on the
similarity between simulated and human basic colour systems (b) the
coordinating role of language on simulated language systems and the
positive impact of language on the similarity between simulated and
human basic colour systems (c) the impact of embodiment on the
performance of different learning operators. In embodied experiments,
two robots perceive a shared environment through their vision
systems. Although this introduces a certain level of noise as both
robots perceive the world from a different perspective, the data
contain a high level of structure as it is based on the colours of
the objects presented to the robots. Overall, embodiment has a
positive effect on the performance of the proposed learning operators.

In the history of a language, a competition between two strategies on
how to express a particular domain might arise. In the domain of
colour, this has been observed in a vast number of languages which
shift from being brightness based to being hue based. The colour term
``yellow'' used to reflect the meaning ``to shine'' in Old English but
shifted to a hue sense in Middle English and could be used to refer to
the colour of yolk or discoloured paper. I present a model in which a
population of agents successfully aligns on which language strategy
they use based on linguistic interactions. I show that this model is
capable of reproducing the meaning shifts similar to those reported in
literature.

Finally, I address some questions on the origins of new language
strategies. New semantic templates can be generated through a
combinatorial search process in which semantic primitives are combined
to form complex semantic templates. I show that each of the proposed
language strategies for the domain of colour can be the outcome of
such a search process. The syntactic templates that have been
introduced to express these templates in language can be incorporated
in repair strategies which allow agents to invent, acquire and align
their own set of grammatical rules. I demonstrate how these repair
strategies allow a population of agents to form their own hierarchical
language that includes some recursive rules. These recursive rules
have the benefit of being able to express more complex meaning without
the cost of alignment in the population.

Even though the examples in this book are limited to the domain of
colour, the proposed templates can easily be extended to richer
examples and deployed in other continuous domains. The proposed
transformation processes could be used to name the colours of concepts
that vary in colour, like for example the colours used to describe
wine. Other possible domains include the spatial domain, in which
spatial categories, such as near and far, also exhibit properties of
graded membership which can be made explicit in language (eg. ``very
near'').  The results reported in this book should hence not be
thought of as final but rather as in interesting starting
point for a whole line of research on the origins and evolution of
natural languages.
% \section*{Acknowledgements} 
% \addchap{Acknowledgements}

Much of the research presented in this book could not have been completed without the use of systems and data that were developed by various members of the wonderful teams of both the Artificial Intelligence Laboratory at the Vrije Universiteit Brussel and the Sony CSL Laboratory in Paris.

I would like to thank Michael Spranger and Martin Loetzsch for their tremendous effort in recording data using the Sony humanoid robots. I am also much obliged to Joachim De Beule, Nicolas Neubauer, Pieter Wellens and Remi van Trijp for the development of FCG, and to Wouter Van den Broeck, Simon Pauw, Michael Spranger and Martin Loetzsch for the development of IRL. Some of the experiments on basic colour systems are also indebted to critical scientific input by Tony Belpaeme. 

And, of course, it is hard to imagine any of this work to materialise without the continuous effort and scientific vision of Luc Steels, the director of both labs.

The research reported in this book has been financially supported by a doctoral grant of the Institute for the Promotion of Innovation through Science and Technology in Flanders (IWT-Vlaanderen).
\tableofcontents      
\mainmatter         

%%%%%%%%%%%%%%%%%%%%%%%%%%%%%%%%%%%%%%%%%%%%%%%%%%%%
%%%                                              %%%
%%%             Chapters                         %%%
%%%                                              %%%
%%%%%%%%%%%%%%%%%%%%%%%%%%%%%%%%%%%%%%%%%%%%%%%%%%%%

\include{chapters/introduction}  %add a percentage sign in front of the line to exclude this chapter from book
\chapter{Generativism}
\section{Chomsky}
\subsection{The early Chomsky}
\citet{Chomsky1957} can be considered the seminal\footnote{
From Latin {\em sēminālis}.
} work.
\chapter{Typology}
\section{The early days of typology}
 ...
\section{The 80s}
\subsection{Comrie}
\citet{Comrie1981} provides a good overview of fundamental concepts.

\ea                                              %numbers the example
\langinfobreak{Dutch}{personal knowledge}{}        %example metadata

\gll Dit is een hond \\                          %example source line. Do not forget the final \\
     \textsc{dem.prox} is a dog\\                %example IMT line. Do not forget the final \\
\glt `This is a dog.'                            %example translation line  
\z                                               %closes the example 
% This file will not be part of the book until you remove the initial percentage sign in lsp-skeleton.tex %uncomment to include this file in your book
\include{chapters/yetanotherfilename} 
%you can add additional chapters below if you want to 

%%%%%%%%%%%%%%%%%%%%%%%%%%%%%%%%%%%%%%%%%%%%%%%%%%%%
%%%                                              %%%
%%%             Backmatter                       %%%
%%%                                              %%%
%%%%%%%%%%%%%%%%%%%%%%%%%%%%%%%%%%%%%%%%%%%%%%%%%%%%
\backmatter
\bibliography{mybibliography.bib} %change to the name of your bib file
\end{document} 

%%%%%%%%%%%%%%%%%%%%%%%%%%%%%%%%%%%%%%%%%%%%%%%%%%%%
%%%                                              %%%
%%%                  END                         %%%
%%%                                              %%%
%%%%%%%%%%%%%%%%%%%%%%%%%%%%%%%%%%%%%%%%%%%%%%%%%%%%

% you should be able to create a pdf from this file 
% with the following command 
% xelatex lsp-skeleton.tex
% If this does not work, please get in contact with 
% Language Science Press
